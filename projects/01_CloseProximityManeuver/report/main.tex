\documentclass{article}
\usepackage[utf8]{inputenc}

\usepackage{amsthm,amssymb,amsmath}
\usepackage{graphicx}

\newcommand{\NN}{\mathbb{N}}
\newcommand{\ZZ}{\mathbb{Z}}
\newcommand{\RR}{\mathbb{R}}
\newcommand{\QQ}{\mathbb{Q}}
\newcommand{\CC}{\mathbb{C}}

\title{AERO7970 - Trajectory Optimiztion // {\small Project 01}}
\author{Matt Boler}
\date{\today}

\begin{document}

\maketitle

\begin{abstract}
  Words here!
\end{abstract}

%%
\section{Question 1}

\subsection*{Solution}
\begin{proof}
  Write your proof here.
\end{proof}

%%
\section*{Some LaTeX Commands}

Here are some example sentences using LaTeX commands:\\
If $a\equiv 2\pmod{3}$, then $a^2\equiv 1\pmod 3$.\\
If $x$ and $y$ are positive real numbers, the arithmetic mean is $\dfrac{x+y}{2}$ and the geometric mean is $\sqrt{xy}$.\\
The union of two sets is $A\cup B$ and the intersection of two sets is $A\cap B$.\\
Let $(x,y)\in A\times B$.\\
Let $f:\mathbb{R}\to\mathbb{R}$ be defined by $f(x)=x^{2019}$.\\
In set-builder notation,  the set of all odd integers is $\{2k+1\mid k\in\mathbb{Z}\}$.\\
Suppose that
\[1+3+5+\cdots+(2k-1) = k^2.\]
Note that if $a=2k$, then
\begin{align*}
    a^2+3a+5 &= (2k)^2+3(2k)+5 \\
             &= 4k^2+6k+4+1 \\
             &= 2(2k^2+3k+2)+1.
\end{align*}
If $g\circ f$ is surjective, then $g$ is surjective.\\

\end{document}
